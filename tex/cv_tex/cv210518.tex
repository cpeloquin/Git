% Medium Length Professional CV
% LaTeX Template
% Version 2.0 (8/5/13)
%
% This template has been downloaded from:
% http://www.LaTeXTemplates.com
%
% Original author:
% Rishi Shah 
%
% Important note:
% This template requires the resume.cls file to be in the same directory as the
% .tex file. The resume.cls file provides the resume style used for structuring the
% document.
%
%%%%%%%%%%%%%%%%%%%%%%%%%%%%%%%%%%%%%%%%%

%----------------------------------------------------------------------------------------
%	PACKAGES AND OTHER DOCUMENT CONFIGURATIONS
%----------------------------------------------------------------------------------------
\documentclass[9pt, letterpage]{resume} % Use the custom resume.cls style
\usepackage{fontspec}
\setmainfont{Trebuchet MS} 
%\usepackage[T1]{fontenc}
%\usepackage[utf8]{inputenc}
% \usepackage{kpfonts, baskervald}
\usepackage[left=0.75in,top=0.6in,right=0.75in,bottom=0.6in]{geometry} % Document margins
\def\changemargin#1#2{\list{}{\rightmargin#2\leftmargin#1}\item[]}
\let\endchangemargin=\endlist 
\newcommand{\tab}[1]{\hspace{.2667\textwidth}\rlap{#1}}
\newcommand{\itab}[1]{\hspace{0em}\rlap{#1}}
\name{Claude Péloquin} % Your name

\address{25 rue Chateaubriand, Gatineau} % Your address
\address{cpeloquin@fastmail.com $|$ 514-816-7448} % Your phone number and email


\begin{document}
%----------------------------------------------------------------------------------------
%	SECTION SOMMAIRE
%----------------------------------------------------------------------------------------

\begin{rSection}{Sommaire} 
Géographe avec plus de dix ans d’expérience en recherche et analyse sur les aspects institutionnels et politiques de la gestion des ressources et de l’aménagement du territoire.
\end{rSection}

%----------------------------------------------------------------------------------------
%	EDUCATION SECTION
%----------------------------------------------------------------------------------------

\begin{rSection}{Formation} 

\textbf{Doctorat en géographie}, Université de l'Arizona \\ 
\textbf{Maîtrise en gestion des ressources naturelles}, Université du Manitoba \\ 
\textbf{Baccalauréat en géographie}, Université McGill
\end{rSection}
%----------------------------------------------------------------------------------------
%	WORK EXPERIENCE SECTION
%----------------------------------------------------------------------------------------

\begin{rSection}{Expériences Professionnelles}
  \vspace{0.25em}
\begin{rSubsection}{Analyste principal des politiques}{2018 - présent}{Gouvernement du Canada, ministère des Ressources naturelles}{Ottawa}
\item Recherche et formulation de conseils en soutien à la négociation et à la mise en œuvre des ententes entre le gouvernement fédéral et les collectivités autochtones.
 \end{rSubsection}
 
\begin{rSubsection}{Chercheur en aménagement du territoire et en gestion des ressources}{2017 - 2018}{Gouvernement de la nation crie (Eeyou Istchee)}{Montréal}
\item 
Recherche et analyse sur les enjeux de développement économique et social, de gestion de la faune, et de protection de la biodiversité dans la région Eeyou Istchee Baie-James.
 \end{rSubsection}
 
\begin{rSubsection}{Enseignant en géographie environnementale}{2014 - 2017}{Université Concordia, Macalaster College et Université Concordia}{Tucson, Saint Paul et Montréal}
\item Préparer et enseigner cours en géographie humaine de l'environnement pour premier cycle universitaire.
\item Cours enseignés : \emph{Géographie humaine des enjeux globaux}; \emph{Géographie des risques environnementaux}; \emph{Environnement et société}; \emph{Environnement et développement}; et \emph{Méthodes de recherche en géographie humaine}.
\item Animation de groupes de lecture d'étudiants aux cycles supérieurs (Arizona)  \item Membre de comité d’éthique en recherche en sciences sociales de l’université et supervision d’étudiants en recherche de mémoire (Macalester).	
\end{rSubsection}

\begin{rSubsection}{Chercheur au doctorat en géographie environnementale}{2010 - 2014}{Université de l’Arizona, École de géographie et du développement}{Tucson}
\item Recherche sur les aspects institutionnels des programmes de contrôle des criquets pèlerins (insectes ravageurs migratoires) au Sahel et en Afrique de l’Ouest. 
\item Mener étude combinant recherche historique, entrevues, et séjours de terrain avec professionnels techniques et scientifiques en Mauritanie, au Mali, au Sénégal, au Maroc et en France.
\end{rSubsection}

\begin{rSubsection}{Chercheur invité en relations nature-société}{2010 - 2012}{Cirad, Unité de recherche Gestion des ressources renouvelables et environnement}{Montpellier}
\item Recherche sur les aspects institutionnels du contrôle des insectes ravageurs : les aspects institutionnels du contrôle des mouches blanches (\emph{Bemisia tabaci}) dans les cultures commerciales de tomates en France et du contrôle des criquets pèlerins en Afrique de l'Ouest.
\end{rSubsection}

\begin{rSubsection}{Analyste en environnement}{2008 - 2009}{Comité consultatif pour l’environnement de la Baie James}{Montréal}
\item Recherche sur divers aspects sociaux et environnementaux du développement des ressources dans la région Eeyou Istchee Baie-James.
\end{rSubsection}

\begin{rSubsection}{Recherche à la maîtrise et assistant de recherche}{2005 - 2008}{Université du Manitoba}{Winnipeg}

\item Recherche sur le rôle du savoir écologique des chasseurs cris dans leurs réponses aux changements sociaux et environnementaux entourant la chasse à l’oie dans à Wemindji, au nord du Québec.

\item Dans le cadre de projet menant à la réserve de biodiversité projetée Paakumshumwaau-Maatuskaau (Université McGill), assurer appui logistique aux études de terrain, cartographie, contribution à la rédaction collective de rapports techniques et révision des travaux collectifs concernant l’écologie, culture et développement dans la rions Eeyou Istchee-Baie-James.
\end{rSubsection}
\end{rSection}

\begin{rSection}{Autres activités} 
Bénévole en soutien à la francisation au Comité de soutien aux réfugiés syriens et irakiens accueillis à Gatineau. 

Récipiendaire de bourses de la National Science Foundation (USA), de l'Arizona Science Foundation, du Conseil de recherche en sciences humaines du Canada, et du Cirad (France) et travaux de recherche récompensés par prix de l'association américain des géographes. 

Travaux de recherches publiés dans les revues scientifiques \emph{Geoforum}, \emph{Global Environmental Change} et \emph{Human Ecology} ainsi que dans ouvrages collectifs publiés par presses universitaires. 




\end{rSection}
\end{document}
 
 
