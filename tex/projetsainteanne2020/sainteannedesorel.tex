\documentclass[12pt]{report}
*\usepackage[utf8]{inputenc}
*\usepackage[T1]{fontenc}
*\usepackage{kpfonts,baskervald}
*\usepackage[french]{babel}
\begin{document}
\chapter*{Prologue}

On fait tourner un globe terrestre et on pointe à un endroit au hasard. Ou l'on fait l'équivalent sur \emph{Google Earth}. Au hasard. Et qui sait ce qui en suivra. C'est un peu comme ça que j'ai choisi de centrer ce livre sur ce lieux: parce que j'y suis né et que j'ai grandi, et c'est de là que ma compréhension du monde vient. Cette ancre fait que \og le reste \fg, pour moi, ne me fait jamais autant de sens, comme diraient les Anglais : sans ce cadre de références ma compréhension des autres endroits que j'apprends à connaître me semble juste juste \og flotter \fg  comme une série de faits dont la trame narrative les assemblant, bien que conceptuellement tangible, reste insaisissable au delà d'un exercice purement cérébral. Ces autres lieux sont, en soi, bien sûr vrais, mais un peu moins vrais que là.  

À titre de géographe, j'ai étudié des dynamiques socio-environnementales particulières dont la sélection comme projet de recherche était avant tout issue d'un intérêt théorique partagé par mes mentors, les participants, les agences de financement, et moi même, surtout mon goût d'apprendre comment le monde marche, ou du moins, une petite partie de celle-ci, tel que vu par la lentille d'un cas particulier qui anime et met en lumière des processus que je cherche à mieux comprendre. Mais alors qu'un chercheur universitaire en sciences humaines à cette conjoncture du XXIe siècle doit souvent faire rouler son produit de recherche de façon à mousser les citations et la notoriété des concepts innovateurs dont il vaudra être reconnu comme l'auteur, mon parcours en fut plutôt d'interrogations spontanées que je vois plutôt comme cul-de-sac dans le sens le moins péjoratif du terme, s'il en est un. Pas des échecs, des explorations conceptuelles qui ont une fin, voilà tout. Celles-ci ont inclus la description d'aspects de la chasse aux oies migratoirepar les chasseurs de la nation crie Eeyou Istchee dans la région des terres cries de la baies au nord du québec, ainsi que plusieurs autres questions, notamment les caribous et l'aménagement du territoire. Les Organisation non-gouvernementales au Bangladesh dans les réseaux internationaux de philantropie et aide au développement. Plus récemment, le dispositif de gestion du criquet pèlerin en Afrique de l'Ouest, et, parce que ces activités ci-nommées, bien que beaucoup plus interessantes, n'ont malheureusement pas toujours garanti des revenus suffisants, 
 


\chapter{Le Pot-au-beurre: l'histoire d'un ruisseau}
Dans l'histoire d'un ruisseau, Élisée Reclus
Ce ruisseau est l'abouttissement de processus géographiques, 




\chapter{Le bout du monde}
%\section{Section}
% \subsection{Une sous-section}
Sorel, maintenant Sorel-Tracy, depuis la fusion à la suite de laquelle les résidents des deux principales villes formant cette agglomération on choisi de renommer cette ville \og fondée \fg il y a près de 400 ans -- en 1642, même année Montréal -- et le centre industriel, commercial et politique du bas-Richelieu, en diminuant son statut autant d'importance toponymique à sa banlieue d'au combien secondaire, fondée seulement en 1954, pratiquement une company-town. 

Comme ailleurs, Trois-Rivière, Shawinigan, Gatineau, ou Rouyn-Noranda, l'histoire de Sorel est un condensé de l'histoire du Québec, dans toute ces contradictions culturelles, économiques, politiques. 

C'est aussi un cul-de-sac. Probablement la principale source de sa notoriété. Depuis que l'autoroute 30 fût rallongé pour croiser les autres, 20, 15, 10 et rejoindre la 40, Sorel-Tracy figure sur les paneaux de signalisation de presque tous les raccord et carrefours, avec Montréal, Toronto, Ottawa-Gatineau, et Québec. Que

Sorel est au bout de l'autre. Et 


\chapter{Les basses-terres du Saint-Laurent}

Dépôt alluvial, voie de transport historique. Dépot des îles. 

\chapter{Les premiers habitants}

Les Indiens que l'on peut connaître, le seul rappel que les Français sont arrivés ici sur une terre jadis peuplée est la réserve des abénaquis d'Odanak, eux-mêmes relocalisés depuis ce qui est maintenant le Connecticut. 

\chapter{Les français}


\chapter{Les Anglais}

Les notables et proto-capitalistes anglais sont partis depuis longtemps. La \og Maison des gouverneurs \fg, l'église Anglicane le carré central et le nom des rues centrales données aux représentants de la monarchie britannique en sont les seuls souvenirs. Leur successeurs francophones y ont mené une partie important de l'industrialisation du Québec (voir ci-bas). 


\chapter{Les champs}


\chapter{Les usines}

Les chantieres navals de la famille Simard et Marine Industries, et les fonderies, Crucible Steel, Beloit Steel et autres. Mon grand-père Péloquin a travaillé 39 ans dans le premier, mon Grand-père plante à peu près aussi longtemps dans le deuxième. 

\end{Document}
