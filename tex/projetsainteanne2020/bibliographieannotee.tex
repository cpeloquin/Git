\documentclass[12pt]{article}
\usepackage[utf8]{inputenc}
\usepackage[T1]{fontenc}
\usepackage{kpfonts,baskervald}
\usepackage{lipsum}  
\usepackage[french]{babel}
\usepackage{csquotes}
%\setlength{\parindent}{1em}
\setlength{\parskip}{0.5em}
\usepackage[backend=biber, style=iso-authoryear]{biblatex}
\DefineBibliographyStrings{french}{in={dans},inseries={dans}}
\bibliography{bib1.bib}
\title{Le Chenal-du-Moine : bibliographie annotée}
\author{Claude Péloquin}
\date{}
\begin{document}
\maketitle
\section{Essai géographique}

\noindent
\fullcite{Reynaud1990}.

Je me rend compte à quel point j'ai ignoré cette tradition humaniste de la géographie culturelle, je n'avais \og pas le temps pour ça \fg parce que \og ça sert à rien \fg. 

\noindent
\fullcite{Guevremont1990}

Pas sûr de ce que j'ai à dire la-dessus mais c'est sûr que je dois le mentionner, c'est vraiment par là que la petite place que Sainte-Anne a pu avoir dans l'imaginaire au delas du lieu-dit est passée. 

\noindent
\fullcite{DeKonninck2000}

La base, probablement une bonne partie de ce qui a mis l'idée du métier de géographer dans ma tête. 

\end{document}



Ce texte est à la fois en aval et en parallèle aux textes suivants, à divers degrés. 
\subsubsection*{Sources sur la géographie de Sorel et des environs}

\begin{itemize}
	\item Les Cent-Îles du Lac Saint-Pierre : retour aux sources et nouveaux enjeux, Rodolphe De Koninck, 2000.
	\item Le Chenal du Moine: une histoire illustrée à l'occasion du centenaire de la paroisse et de la municipalité de Sainte-Anne de Sorel, 1876-1976, Walter S. White, 1980.
	\item Histoire de Sorel de ses origines à nos jours, Azarie Couillard-Després, 1926.
\end{itemize}

\subsubsection*{Sources théoriques et inspirations}
\begin{itemize}
	\item L’Homme et la Terre, Élisée Reclus, 1905.
	\item L'histoire d'un ruisseau, Élisée Reclus, 1869.
	\item Les Terres, Louis-Nicholas Trépannier, 2019.
\end{itemize}


%sources à lire: 
% https://www.google.com/url?sa=t&rct=j&q=&esrc=s&source=web&cd=7&cad=rja&uact=8&ved=2ahUKEwi83vyvivroAhX2hHIEHYccA_wQFjAGegQIBBAB&url=https%3A%2F%2Fwww.erudit.org%2Ffr%2Frevues%2Fcgq%2F2001-v45-n124-cgq2696%2F022963ar%2F&usg=AOvVaw2wYq33lGWRs8HOAJaKtfFq
% Élisée Reclus, 