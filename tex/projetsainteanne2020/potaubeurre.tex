\documentclass[12pt]{report}
\usepackage[utf8]{inputenc}
\usepackage[T1]{fontenc}
% \usepackage{kpfonts,baskervald}
\usepackage[french]{babel}
\usepackage[backend=biber, style=iso-authoryear]{biblatex}
\DefineBibliographyStrings{french}{in={dans},inseries={dans}}
\bibliography{bib1.bib}
\title{Le Chenal-du-Moine : bibliographie annotée}
\author{Claude Péloquin}
\date{}
\begin{document}
\chapter*{Projet parallèle : la Rivière Pot-au-Beurre}

\maketitle
\section{Introduction}

Le premier cours d'eau que j'ai connu c'est le rang des vingts, pour la décharge des vingts arpents, un peu plus loin il y a les quarantes. Derrière chez nous c'était juste avant l'entrée du bois. Le long du boisé, les cinquante, et puis, au delà de ça, la rivière du pot-au-beurre. Ce nom et c'est endroit m'a toujours intrigué. Les autres cours d'eau pouvaient être traversés par des ponts puisqu'elles reliaient des terres cultivées, souvent appartenant aux mêmes personnes. Le pot-au-beurre, c'était la limite absolue. À notre longitude, ses alentours étaient marécageux. 

Les arpents, une mesure désuète en France mais qui est demeurée dans les colonies françaises. \og French Acres \fg on dit en anglais.
Pot au beurre: 

mené d'herbe
Description du projet

La Rivière Pot au Beurre est située dans le bassin versant de la rivière Yamaska, lequel rejoint le fleuve Saint-Laurent au niveau du lac Saint-Pierre. L’uniformité paysagère agricole, la gestion minimaliste des bandes riveraines et les pratiques agricoles sont autant de causes reliées à la mauvaise qualité de l’eau du bassin versant. Le ruisseau Saint-Thomas et Sainte-Sophie, lui, est un cours d’eau tributaire de la Rivière Pot au Beurre. Le méné d’herbe que l’on retrouve en aval du ruisseau ne remonte plus celui-ci. Pourtant, ce cours d’eau à méandres offre potentiellement un milieu propice à l’espèce. Ce projet, porté par la Fédération de l’UPA de la Montérégie, vise à restaurer un secteur identifié afin de le rendre plus favorable au méné d’herbe. Pour ce faire, le projet prévoit d’effectuer des aménagements de stabilisation et de végétalisation de berge.
Objectifs 

    Augmenter la population du méné d’herbe du bassin versant de la rivière Yamaska et du fleuve Saint-Laurent.
    Améliorer l’habitat du méné d’herbe dans le bassin versant de la Rivière Pot au Beurre.
    Réduire les apports de sédiments et rétablir la circulation du méné d’herbe dans la Rivière Pot au Beurre.

Retombées attendues

    Réaliser une étude hydraulique pour identifier les travaux riverains.
    Stabiliser les berges en érosion.
    Végétaliser des talus des rives.
    Planter des haies brise-vent et des îlots arbustifs.
    Préserver le profil de la rive droite pour favoriser le dépôt des sédiments et l’apparition d’herbiers.




https://robvq.qc.ca/affluents_maritime/upa

\section{Géographie}